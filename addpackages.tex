% =========================================================
% Inclusão e configuração de pacotes extras
% =========================================================


% siunitx =================================================

\usepackage{siunitx}

\sisetup{
    number-mode=match,
    unit-mode=text,
    parse-numbers=true,
    % round-mode=places, round-precision=4,
    % round-zero-positive=false,
    % retain-explicit-plus=true,
    % retain-negative-zero=true,
    retain-zero-uncertainty=true,
    exponent-mode=input,
    output-decimal-marker=\text{,},
    exponent-product={\cdot},
    uncertainty-mode=separate,
    per-mode=symbol,
    sticky-per,
    group-digits=integer,
    group-minimum-digits=5,
    group-separator={\,},
    list-separator={; },
    list-final-separator={; e },% Language sensitive
    list-pair-separator={ e },% Language sensitive
    list-units=repeat,
    product-mode=symbol,
    product-symbol={\cdot},
    product-units=repeat,
    range-phrase={ a },% language sensitive
    range-units=repeat,
    %output-complex-root={\imath},% <-- Complex as 'i'
    output-complex-root={\jmath},% <-- Complex as 'j'
    complex-root-position=before-number,
    angle-mode=decimal,
    quantity-product={\ },
    tight-spacing=true,
}

% xcolor ==================================================

\usepackage[
    luatex,
    rgb,
    % cmyk,
    % gray,
    table,
]{xcolor}

\definecolor{codebackcolor}{HTML}{F8F8F8}

% listings ================================================

\usepackage{listings}

\lstset{
    % backgroundcolor=\color{codebackcolor},
    basicstyle=\ttfamily\small,
    numbers=none,
    % numberstyle=\ttfamily\small,
    % numbersep={10pt},
    breakatwhitespace=false,
    breaklines=true,
    keepspaces=true,
    showspaces=false,
    showstringspaces=false,
    showtabs=false,
    tabsize=2,
    frame=single,
    captionpos=b,
    abovecaptionskip={6pt},
    belowcaptionskip={6pt},
}

% Changes caption and list of listings names (language senbsitive)
\renewcommand{\lstlistingname}{Programa}
\renewcommand{\lstlistlistingname}{Lista de Programas}

% Defines the apperancer of list of listings
\makeatletter
\AtBeginDocument{%
\renewcommand\lstlistoflistings{\bgroup
  \let\contentsname\lstlistlistingname
  \def\l@lstlisting##1##2{\@dottedtocline{1}{0em}{3em}{##1}{##2}}
  \let\lst@temp\@starttoc \def\@starttoc##1{\lst@temp{lol}}%
  \tableofcontents \egroup}
}
\makeatother

% svg =====================================================

\usepackage{svg}

% tabularx ================================================

\usepackage{tabularx}

\renewcommand{\arraystretch}{1.0}
\renewcommand{\tabularxcolumn}[1]{>{\small}m{#1}}
\newcolumntype{C}{>{\centering\arraybackslash}X}
\newcolumntype{L}{>{\raggedright\arraybackslash}X}
\newcolumntype{R}{>{\raggedleft\arraybackslash}X}

% multirow ================================================

% \usepackage{multirow}

% pdflscape ===============================================

% \usepackage[luatex]{pdflscape}

% pdfpages ================================================

% \usepackage{pdfpages}